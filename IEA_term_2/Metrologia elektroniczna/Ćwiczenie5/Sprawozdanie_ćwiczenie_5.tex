
% Preamble
\documentclass[11pt]{article}

% Packages
\usepackage{amsmath}
\usepackage{mathtools}
\usepackage{ragged2e}
\usepackage [utf8]{inputenc}
\usepackage{blindtext}
\usepackage{wrapfig}
\usepackage{xcolor}
\usepackage {polski}
\usepackage{multicol}
\usepackage[a4paper, total={5.7in, 8in}]{geometry}
\usepackage{graphicx}
\usepackage{amstex}
\usepackage{csvsimple}
\usepackage{changepage}
\usepackage{enumitem}
\usepackage[english]{babel}
\usepackage{biblatex}
\usepackage{caption}
\usepackage{indentfirst}
\usepackage{epstopdf-base}

% Document
\begin{document}
%    Nagłówek
    \begin{flushleft}
        Filip Krauz-Damski 267 681 \hfill Data wykonania ćwiczenia:\\
        Filip Kubecki 272 655 \hfill 15 kwietnia 2024r\\
        \hfill Data sporządzenia sprawozdania:\\
        Grupa: Pon 13:15 \hfill 20 kwietnia 2024r\\
    \end{flushleft}
    \begin{center}
        \Large\textbf{Ćwiczenie 5.}\\
        \textbf{Rejestracja i wyznaczanie parametrów
        sygnałów okresowo zmiennych}
    \end{center}
    \vspace{1cm}

%    Treść
    \section{Spis przyrządów}
    \par{
        Do wykonania ćwiczenia wykorzystano:
        \begin{itemize}
            \setlength\itemsep{0em}
            \item[-] Przenośny multimetr cyfrowy AX-588B
            \item[-] Przenośny multimetr cyfrowy AX-MS8221A
            \item[-] Multimetr cyfrowy Agilent 34401A
            \item[-] Oscyloskop cyfrowy Agilent DSO3062A
            \item[-] Generator funkcji Agilent 33220A
        \end{itemize}
    }
    \section{Przebieg i cele doświadczeń}
    Doświadczenie polegało kolejno na:
    \begin{itemize}
        \setlength\itemsep{0em}
        \item Pomiarze wartości skutecznej napięcia dla sygnałów o różnym kształcie oraz różnej częstotliwości,
        \item Zestawieniu układu pomiaru napięcia oraz natężenia zmiennego na rezystorze oraz zbadaniu czy dla przebiegów okresowo zmiennych zachodzi prawo Ohma,
        \item Pomiarze częstotliwości oraz okresów sygnałów okresowo zmiennych,
        \item Pomiarze przesunięcia fazowego na układzie opóźniającym,
    \end{itemize}
    \newpage

    \section{Wyniki pomiarów}
    \subsection*{Część 1 - Pomiar napięć zmiennych}
    \begin{center}
        \small{\textbf{Tabela 1 - AX-MS8221A funkcja Sinus}}
    \end{center}
    \begin{center}
        \csvreader[tabular = |c|c|c|c|c|,
            table head = \hline  \textbf{Częstotliwość[Hz]} & \textbf{$U_{opt}$[V]} & \textbf{$U_{opt+1}$[V]} & \textbf{$U_{scope}$[V]} & \textbf{$U_{RMS}$[V]} \\\hline,
%            table foot = \hline,
            late after line = \\\hline
        ]{Dane/Dane1AXMS.csv}{}{
            \csvcoli & \csvcolii & \csvcoliii & \csvcoliv & \csvcolv
        }
    \end{center}

    \begin{center}
        \small{\textbf{Tabela 2 - Agilent 34401A funkcja Sinus}}
    \end{center}
    \begin{center}
        \csvreader[tabular = |c|c|c|c|c|,
            table head = \hline  \textbf{Częstotliwość[Hz]} & \textbf{$U_{opt}$[V]} & \textbf{$U_{opt+1}$[V]} & \textbf{$U_{scope}$[V]} & \textbf{$U_{RMS}$[V]} \\\hline,
%            table foot = \hline,
            late after line = \\\hline
        ]{Dane/Dane1Agilent.csv}{}{
            \csvcoli & \csvcolii & \csvcoliii & \csvcoliv & \csvcolv
        }
    \end{center}

    \begin{center}
        \small{\textbf{Tabela 3 - AX-588V funkcja Sinus}}
    \end{center}
    \begin{center}
        \csvreader[tabular = |c|c|c|c|c|,
            table head = \hline  \textbf{Częstotliwość[Hz]} & \textbf{$U_{opt}$[V]} & \textbf{$U_{opt+1}$[V]} & \textbf{$U_{scope}$[V]} & \textbf{$U_{RMS}$[V]} \\\hline,
%            table foot = \hline,
            late after line = \\\hline
        ]{Dane/Dane1AX588.csv}{}{
            \csvcoli & \csvcolii & \csvcoliii & \csvcoliv & \csvcolv
        }
    \end{center}

    \begin{center}
        \small{\textbf{Tabela 4 - Funkcja: piła - Częstotliwość: 1[kHz]}}
    \end{center}
    \begin{center}
        \csvreader[tabular = |c|c|c|c|c|,
            table head = \hline  \textbf{Miernik} & \textbf{$U_{opt}$[V]} & \textbf{$U_{opt+1}$[V]} & \textbf{$U_{scope}$[V]} & \textbf{$U_{RMS}$[V]} \\\hline,
%            table foot = \hline,
            late after line = \\\hline
        ]{Dane/Dane1Tri.csv}{}{
            \csvcoli & \csvcolii & \csvcoliii & \csvcoliv & \csvcolv
        }
    \end{center}

    \begin{center}
        \small{\textbf{Tabela 5 - Funkcja: prostokąt - Częstotliwość: 1[kHz]}}
    \end{center}
    \begin{center}
        \csvreader[tabular = |c|c|c|c|c|,
            table head = \hline  \textbf{Miernik} & \textbf{$U_{opt}$[V]} & \textbf{$U_{opt+1}$[V]} & \textbf{$U_{scope}$[V]} & \textbf{$U_{RMS}$[V]} \\\hline,
            late after line = \\\hline
        ]{Dane/Dane1Sqr.csv}{}{
            \csvcoli & \csvcolii & \csvcoliii & \csvcoliv & \csvcolv
        }
    \end{center}

    \subsection*{Część 2 - Obwody prądu zmiennego}
    \begin{center}
        \small{\textbf{Tabela 6}}
    \end{center}
    \begin{center}
        \csvreader[tabular = |c|c|c|c|c|c|,
            table head = \hline  \textbf{Funkcja} & \textbf{Rezystor[$\Omega$]} & \textbf{U[V]} & \textbf{I[mA]} & \textbf{\boldmath$\frac{U}{I}$[$\Omega$]} & \textbf{\boldmath$u_c(\frac{U}{I})$[$\Omega$]} \\\hline,
            late after line = \\\hline
        ]{Dane/Dane2.csv}{}{
            \csvcoli & \csvcolii & \csvcoliii & \csvcoliv & \csvcolv & \csvcolvi
        }
    \end{center}

    \subsection*{Część 3 - Pomiar częstotliwości i okresu}
    \begin{center}
        \small{\textbf{Tabela 7}}
    \end{center}
    \begin{center}
        \csvreader[tabular = |c|c|c|c|c|c|c|,
            table head = \hline  \textbf{$f$[Hz]} & \textbf{$f_{agil}$[Hz]} & \textbf{$T_{agil}$[ms]} & \textbf{$u(f_{agil})$[Hz]} & \textbf{$u(T_{agil})$[ms]} & \textbf{$f_{588B}$[Hz]} & \textbf{$u(f_{588B})$[Hz]} \\\hline,
            late after line = \\\hline
        ]{Dane/Dane3.csv}{}{
            \csvcoli & \csvcolii & \csvcoliii & \csvcoliv & \csvcolv & \csvcolvi & \csvcolvii
        }
    \end{center}

    \subsection*{Część 4 - Pomiar przesunięcia fazowego}
    \begin{center}
        Sygnał 1[kHz] - przesunięcie fazowe: 132[$\mu$s]
    \end{center}
    \newpage
    \section{Analiza wyników}
    \subsection*{Część 1 - Pomiar napięć zmiennych}
    Przy pomocy wszystkich mierników na stanowisku mierzono wartości napięcia RMS (Root Mean Square — średnia kwadratowa/wartość skuteczna),
    kolejnych przebiegów. Na początku warto zaznaczyć zakresy poprawnego pomiaru wartości RMS napięcia zmiennego dla kolejnych miernikó:
    \begin{itemize}
        \item Agilent 34401A - od 3[Hz] do 300[kHz],
        \item Axiomet AX-588B - od 40[Hz] do 400[Hz],
        \item Axiomet AX-MS8221A - od 40[Hz] do 1[kHz],
    \end{itemize}
    \par Możemy od razu zauważyć, że dla pomiarów napięcia RMS sygnałów sinusoidalnych z zakresu 5[Hz] do 500[kHz] mierniki nie są w stanie zmierzyć wartości dla wszystkich sygnałów.
    Najgorzej plasuje się miernik AX-588B, dla którego jedyny pomiar, jaki mieści się w jego zakresie pomiarowym to pomiar przy częstotliwości 100 [Hz]. Najlepiej w tym zestawieniu
    pokazuje się miernik Agilent 34401A, dla którego jedyny pomiar, który wychodzi poza zakres pomiaru to pomiar przy częstotliwości 500 [kHz]. Miernik ten jednak pozwala na pomiary
    do częstotliwości nawet 1[MHz] jednak przy tej wartości producent zaleca do przyjąć wartość niepewności na aż 30\%. Producent również nie podaje dokładnych niepewności dla pomiarów
    powyżej 300[kHz].\\
    \indent Wartość mierzona w tym doświadczeniu to wartość skuteczna napięcia RMS. Była ona mierzona dla trzech typów sygnałów: sinusoidalnego, piłokształtnego oraz prostokątnego.
    Wartości poprawnych napięć zostały wyliczone korzystając z napięcia międzyszczytowego zmierzonego przy pomocy oscyloskopu (oznaczonego w tabelach jako $U_{scope}$). Wartość ta
    została podstawiona do wzorów na wartość skuteczną napięcia kolejnych sygnałów. Wzory te zostały podane poniżej:
    \subsubsection*{1. Sygnał Sinusoidalny}
    \begin{gather*}
        U_{RMS}=\frac{U_{pp}}{2sqrt{2}}\approx U_{pp}\cdot 0.35355\dots
    \end{gather*}
    Przykładowo dla sygnału sinusoidalnego o wartości międzyszczytowej 20.2[V]:
    \begin{gather*}
        U_{RMS}=20.2[V] \cdot 0.35355\dots=7.142\dots[V]
    \end{gather*}

    \subsubsection*{2. Sygnał Piłokształtny}
    \begin{gather*}
        U_{RMS}=\frac{U_{pp}}{2\sqrt{3}}\approx U_{pp}\cdot 0.2886\dots
    \end{gather*}
    Przykładowo dla sygnału piłokształtnego o wartości międzyszczytowej 20.2[V]:
    \begin{gather*}
        U_{RMS}=20.2[V] \cdot 0.2886\dots=5.831\dots[V]
    \end{gather*}

    \subsubsection*{3. Sygnał Prostokątny}
    \begin{gather*}
        U_{RMS}=U_{pp}\cdot 0.5
    \end{gather*}
    Przykładowo dla sygnału prostokątnego o wartości międzyszczytowej 20.4[V]:
    \begin{gather*}
        U_{RMS}=20.4[V] \cdot 0.5=10.2[V]
    \end{gather*}

    \noindent Wartości skuteczne zostały umieszczone w ostatniej kolumnie tabel 1 - 5.\\
    \par Dla pomiarów nie zostały wyznaczone niepewności napięcia skutecznego ponieważ do poprawnej analizy należałoby wziąć pod uwagę wszystkie elementy układu takie jak:
    niepewność miernika, niepewność odczytu oscyloskopu, impedancje dodane przez urządzenia pomiarowe, niepewności wynikająca z stabilizowania się wyników.\\
    \indent Można zaobserwować że dla pomiarów dokonanych w zakresie operowania multimetrów zauważalna jest tendencja do zaniżania wyniku pomiaru dla zakresu o jeden wyższego niż optymalny.
    Jedynie w przypadku miernika Agilent 34401A różnica między wynikiem na zakresie optymalnym a zakresem o jeden wyższym nie wykazuję tej tendencji. Wynika to prawdopodobnie z o
    wiele bardziej zaawansowanego układu pomiarowego miernika.\\
    \indent Mierniki wartości skutecznej dzielimy na mierniki True RMS oraz mierniki które wyliczają wartość skuteczną na podstawie wartości maksymalnej przebiegu rejestrowanego.
    Jedyny miernik który nie jest miernikiem nie będącym miernikiem True RMS był miernik AX-MS8221A. Zgodnie z notą katalogową miernik ten oblicza wartość skuteczną dla sygnału
    sinusoidalnego. Miernik ten powinien więc pokazywać poprawne wskazania dla sygnału sinusoidalnego jednak dla sygnału piłokształtnego oraz prostokątnego powinien pokazywać
    błędne wyniki.\\
    \indent Gdy porównamy wartości zmierzone sygnału piłokształtnego oraz prostokątnego do wartość RMS obliczonych na podstawie wartości międzyszczytowej zmierzonej oscyloskopem zauważamy
    że wszystkie pomiary są zgodne w wartościami oczekiwanymi. Jednak miernik AX-MS8221A powinien wskazywać takie same wyniki jak dla sygnału sinusoidalnego gdyż nie mierzy on rzeczywistej
    wartości skutecznej napięcia. Nie jesteśmy w stanie wyciągnąć dokładnego wzniosku dlaczego miernik ten pokazuje poprawne wyniki mimo iż wykorzystuję uproszczoną metodę. Aby lepiej
    zrozumieć przyczyny tego zjawiska potrzebne byłyby dane na temat dokładnej metody obliczania wartości RMS przez ten miernik. Danych takich jednak producent nie podaje.
    \newpage
    \subsection*{Część 2 - Obwody prądu zmiennego}
    Wykonano pomiary dwóch rezystancji (100 [$\Omega$] i 1000 [$\Omega$) metodą poprawnego pomiaru napięcia przy zasileniu układu napięciem zmiennym
    o częstotliwości 1 [kHz] i o kształcie funkcji sinudoidalnej, piłokształtnej oraz prostokątnej. Mimo zastosowania metody poprawnego pomiaru napięcia w obliczeniach
    nie uwzględniano prądu bocznikowanego przez woltomierz gdyż rezystancja wewnętrzna woltomierza (wynoszoąca 10[M$\Omega$]) była tak duża że wpływ na ostateczny wynik
    był pomijalnie mały.\\
    \indent Badanie miało za zadanie sprawdzić czy układy napięcia zmiennego zachowują prawo Ohma. Mierzono więc napięcie RMS na rezystorze oraz prąd wpływający do rezystora.
    Rezystancja została wyliczona przy użyciu prawa Ohma:
    \begin{gather*}
        R=\frac{U}{I}
    \end{gather*}
    Przykładowo dla pomiaru na rezystorze 100 [$\Omega$] dla zasilania funkcją sinus:
    \begin{gather*}
        R=\frac{2.2767[V]}{22.9[mA]}=99.4192\dots [\Omega]]
    \end{gather*}
    \par Na podstawie pomiarów jesteśmy w stanie zauważyć że różnica między pomiarami a rezystancjami podanymi na rezystorach różni się dla niektórych pomiarów poza granicę niepewności pomiarowej.
    Wynika to z nieuwzględnienia w niepewności tolerancji wykonania rezystorów na których były wkonywane pomiary. Płytki z rezystorami wykorzystane w ćwiczeniu
    były już używane w innych ćwiczeniach a ich tolerancje wynosiły 5\%. Po dodaniu tego założenia widać że rezystancje pokrywają się z przewidywaniami w zakresie niepewności oraz tolerancji.
    Wynikałoby z tego że prawo Ohma zachodzi również dla sygnałów zmiennych.\\
    \indent Po zapoznaniu się z teorią widzimy że dla napięcia zmiennego prawo Ohma wyraża się wzorem:
    \begin{gather*}
        I=\frac{U}{Z}
    \end{gather*}
    \indent Możemy zauważyć że jest to praktycznie wzór jak dla prądu stałego jednak rezystancję R zastępuje nam impedancja Z. W przypadku naszego układu impedancja mierzona równa jest rezystancji
    gdyż nie posiadamy elementów które posiadałyby reaktancję.
    
    \subsection*{Część 3 - Pomiar częstotliwości i okresu}
    W ćwiczeniu zmierzono przy pomocy miernika Agilent 34401A (pomiary przy jego użyciu zaznaczone zostały w tabeli 7 przypisem $agil$) oraz AX-588A (oznaczonego przypisem $588A$)) zmierzono
    częstotliwości sygnału sinusoidalnego o napięciu międzyszczytowym 1 [V]. Pomiary obejmowały częstotliwości od 1 [Hz] do 500 [kHz]. Poniżej przedstawiono zakresy pomiarowe wykorzystanych
    mierników:
    \begin{itemize}
        \item Agilent 34401A - 3 [Hz] do 300 [kHz] (miernik umożliwia również pomiary poniżej 3 [Hz] jednak producent nie gwarantuje ich dokładności i nie podaje niepewności takich pomiarów),
        \item AX-588B - 1 [Hz] do 10 [MHz],
    \end{itemize}
    \par W tabeli 7 zawierającej wyniki jesteśmy w stanie zauważyć że w przypadku miernika Agilent 34401A nie udało się zmierzyć najwyższej częstotliwości o wartości 500 [kHz]. Jest to spowodowane
    przekroczeniem zakresu pomiarowego urządzenia.\\
    \indent Można zauważyć że wszystkie pomiary mieszczą się w niepewnościach pomiarowych mierników. Obrazuje to rzetelność pomiarów wykonywanych miernikami. Dodatkowo można zauważyć
    że miernik Agilent 34401A oferuje o wiele dokładniejsze pomiary niż miernik AX-588B. Miernik AX-588B jednak wygrywa szerokością zakresu pomiarowego która jest znacznie większa od
    tego oferowanego przez miernik Agilent 34401A.\\\\
    \indent Niepewności pomiaru częstotliwości oraz okresu dla miernika Agilent 34401A (obie wartości liczy się analogicznie), wyraża się wzorem:
    \begin{gather*}
        u(f)=\alpha\%\cdot rdg
    \end{gather*}
    {\footnotesize
        \begin{itemize}
            \setlength\itemsep{0em}
            \item[] \boldmath$\alpha$ - współczynnik podany przez producenta,
            \item[] \boldmath$rdg$ - wartość odczytana,
        \end{itemize}}
    \noindent Przykładowo dla pomiaru częstotliwości 500 [Hz]:
    \begin{gather*}
        u(f)=0.006\%\cdot 499.992 [Hz]=0.02999952 [Hz]\approx 0.030 [Hz]
    \end{gather*}
    \noindent Niepewności pomiaru częstotliwości dla miernika AX-588V:
    \begin{gather*}
        u(f)=\alpha\cdot rdg+ c\cdot X_{min}
    \end{gather*}
    {\footnotesize
        \begin{itemize}
            \setlength\itemsep{0em}
            \item[] \boldmath$\alpha$ - współczynnik podany przez producenta,
            \item[] \boldmath$rdg$ - wartość odczytana,
            \item[] \boldmath$c$ - współczynnik podany przez producenta,
            \item[] \boldmath$X_{min}$ -  rozdzielczość zakresu,
        \end{itemize}}
    \noindent Przykładowo dla pomiaru częstotliwości 500 [Hz]:
    \begin{gather*}
        u(f)=0.5\%\cdot 499.992 [Hz]+5\cdot 1 [Hz]=6.49[Hz]\approx 6.5 [Hz]
    \end{gather*}
    \newpage
    \subsection*{Część 4 - Pomiar przesunięcia fazowego}
    Przy pomocy oscyloskopu Agilent DSO3062A wyliczono przesunięcie fazowe sygnału sinusoidalnego o częstotliwości 1 [kHz] oraz wartości międzyszczytowej 1 [V]. Sygnał ten został
    przepuszczony przez ścieżkę o czystej rezystancji oraz przez układ opóźniający składający się dodatkowo z kondensatora. Przesunięcie fazowe zostało wyznaczone na podstawie
    przesunięcia w czasie punktu przecięcia osi x sygnału przesuniętego oraz nieprzesuniętego (w fazie rośnięcia sygnału sinusoidalnego).\\
    \indent Wyznaczone przesunięcie fazowe wynosi: 132 [$\mu$ s].
    Przesunięcie fazowe podajemy w jednostkach kątowych, potrzeba wiec wyliczyć o jaką część okresu sygnału o częstotliwość 1 [kHz] przesunięto sygnał. Należy najpierw obliczyć
    okres sygnału. Wyliczymy go z poniżego wzoru:
    \begin{gather*}
        T=\frac{1}{f}=\frac{1}{1[kHz]}=1 [ms]=1000[\mu s]
    \end{gather*}
    Następnie wyliczamy przesunięcie fazowe ze wzorów:
    \begin{gather*}
        \varphi=360^{\circ}\cdot\frac{\tau}{T}=360^{\circ}\cdot\frac{132 [\mu s]}{1000 [\mu s]}=47.52 [^{\circ}]
    \end{gather*}
    lub dla wyniku w radianach:
    \begin{gather*}
        \varphi=2\pi\cdot\frac{\tau}{T}=2\pi\cdot\frac{132 [\mu s]}{1000 [\mu s]}=0.8293804\dots [rad]
    \end{gather*}

    \section{Uwagi i wnioski}
    Z wykonanych ćwiczeń możemy wyciągnąć wiele wniosków przydatnych przy pracy z sygnałami okresowo zmiennymi. Z części 1 możemy wygciągnać 3 kluczowe wnioski:
    \begin{itemize}
        \item[1.] Należy zawsze dbać o wybranie najbardziej optymalnego zakresu pomiarowego dla pomiaru napięcia skutecznego aby uniknąć zaniżenia wyników,
        \item[2.] Należy zapoznać się z notą katalogową miernika i sprawdzić dla jakich częstotliwości miernik potrafi wyznaczać wartość napięcia skutecznego,
        \item[3.] W przypadku pomiarów sygnałów innych niż sygnały sinusoidalne, sprawdzić w nocie katalogowej czy nasz miernik na pewno oferuje pomiar true RMS,
    \end{itemize}
    \newpage
    \indent Z części 2 możemy empirycznie zaobserwować że dla sygnałów okresowo zmiennych w układach o czystej rezystancji zostaje zachowane prawo Ohma. W ogólnym przypadku prawo Ohma dla
    prądu zmiennego prawo Ohma przyjmuję postać:
    \begin{gather*}
        I=\frac{U}{Z}
    \end{gather*}
    Czyli rezystancja zostaje zastąpiona przez moduł impedancji.\\\\
    \indent Z części 3 ćwiczenia jesteśmy w stanie zauważyć jak różne mierniki oferują różne atuty w przypadku pomiaru sygnałów okresowo zmiennych. Obrazuje nam to że przed przeprowadzeniem
    doświadczenia należy zastanowić się dokładnie czy w danym pomiarze kluczowa jest dla nas dokładność pomiaru czy elastyczność wynikająca z szerokiego zakresu pomiarowego.\\
    \indent Część 4 ćwiczenia obrazuje nam graficznie jaki wpływ na napięcie układu mają elementy posiadające kapacytancję. Przsunięcie fazowe może mieć negatywny wpływ na np.:
    układy analogowe przetwarzające informacje dźwiękowe, może zwiększać straty energii przy przesyle prądu elektrycznego.

    %Bibliografia
    \vfill
    \footnotesize
    \begin{thebibliography}{9}
        \bibitem{texbook1}
        https://wzn.pwr.edu.pl/materialy-dydaktyczne/metrologia-elektroniczna
        \bibitem{texbook2}
        https://lpf.wppt.pwr.edu.pl/pomoce-dydaktyczne.php
        \bibitem{texbook3}
        https://www.falstad.com/
        \bibitem{texbook4}
        https://en.wikipedia.org/wiki/Root-mean-square
        \bibitem{texbook5}
        https://en.wikipedia.org/wiki/Phase-(waves)
        \bibitem{texbook6}
        https://en.wikipedia.org/wiki/Ohm\%27s-law
    \end{thebibliography}

\end{document}