
% Preamble
\documentclass[11pt]{article}

% Packages
\usepackage{amsmath}
\usepackage{mathtools}
\usepackage{ragged2e}
\usepackage [utf8]{inputenc}
\usepackage{blindtext}
\usepackage{wrapfig}
\usepackage{xcolor}
\usepackage {polski}
\usepackage{multicol}
\usepackage[a4paper, total={5.7in, 8in}]{geometry}
\usepackage{graphicx}
\usepackage{amstex}
\usepackage{csvsimple}
\usepackage{changepage}
\usepackage{enumitem}
\usepackage[english]{babel}
\usepackage{biblatex}
\usepackage{caption}
\usepackage{indentfirst}
\usepackage{epstopdf-base}

% Document
\begin{document}
%    Nagłówek
    \begin{flushleft}
        Filip Krauz-Damski 267 681 \hfill Data wykonania ćwiczenia:\\
        Filip Kubecki 272 655 \hfill 22 kwietnia 2024r\\
        \hfill Data sporządzenia sprawozdania:\\
        Grupa: Pon 13:15 \hfill 28 kwietnia 2024r\\
    \end{flushleft}
    \begin{center}
        \Large\textbf{Ćwiczenie 6.}\\
        \textbf{Pomiary temperatury}
    \end{center}
    \vspace{2cm}
%    Treść
    \section{Spis przyrządów}
    \par{
        Do wykonania ćwiczenia wykorzystano:
        \begin{itemize}
            \setlength\itemsep{0em}
            \item[-] Multimetr cyfrowy Agilent 34401A
            \item[-] Stolik temperaturowy (specyfikacja nieznana)
        \end{itemize}
    }
    
    \section{Przebieg i cele doświadczenia}
    Doświadczenie polegało kolejno na:
    \begin{itemize}
        \setlength\itemsep{0em}
        \item Wyznaczeniu charakterystyki napięciowo temperaturowej termopary przez podgrzewanie jej stolikiem temperaturowym,
        \item Wyliczenie temperatury obudowy rezystora HS50 na zmierzonej rezystancji czujnika PT-100 znajdującego się na jego obudowie,
        \item Wyznaczeniu zależności temperatury obudowy rezystora HS50 od mocy wydzielonej przez ten rezystor,
    \end{itemize}

    \section{Wyniki pomiarów}
    \subsection*{Część 1 - Charakterystyka termopary}
    \begin{center}
        \csvreader[tabular = |c|c|,
            table head = \hline  \textbf{T[${^\circ}C$]} & \textbf{U[mV]}  \\\hline,
%            table foot = \hline,
            late after line = \\\hline
        ]{Dane/Dane1.csv}{}{
            \csvcoli & \csvcolii
        }
    \end{center}
    \subsection*{Część 2 - Charakterystyka mocy od temperatury rezystora HS50}
    \begin{center}
        \csvreader[tabular = |c|c|c|c|c|,
            table head = \hline  \textbf{I[A]} & \textbf{U[V]} & \textbf{\boldmath$R_{PT100}$[$\Omega$]} & \textbf{\boldmath$T_{PT100}$[${^\circ}C$]} & \textbf{\boldmath$P_{HS50}$[W]}  \\\hline,
%            table foot = \hline,
            late after line = \\\hline
        ]{Dane/Dane2.csv}{}{
            \csvcoli & \csvcolii & \csvcoliii & \csvcoliv & \csvcolv
        }
    \end{center}
    
    \section{Analiza wyników}
    \subsection*{Część 1 - Charakterystyka termopary}
    Charakterystyka napięcia termopary od temperatury mierzonej została przedstawiona na wykresie nr 1 dołączonym do sprawozdania. Doświadczenie zaczęto od temperatury
    0.7 [${^\circ}C$] ponieważ stolik temperaturowy nie pozwalał na osiągnięcie niższej temperatury.

    \subsection*{Część 2 - Charakterystyka mocy od temperatury rezystora HS50}
   \noindent Moc wydzieloną na rezystorze HS50 obliczono ze wzoru:
    \begin{gather*}
        P=U\cdot I
    \end{gather*}
    {\footnotesize
        \begin{itemize}
            \setlength\itemsep{0em}
            \item[] \textbf{P} - moc wydzielona na rezystorze,
            \item[] \textbf{I} - prąd płynący przez rezystor,
            \item[] \textbf{U} - napięcie na zaciskach rezystora,
        \end{itemize}}
    \noindent Przykład moc wydzielona przy napięciu 1.5 [V]:
    \begin{gather*}
        P=1.5 [V]\cdot 0.65 [A]=0.975 [W]
    \end{gather*}
    \noindent Temperaturę zmierzoną termoparą PT100 wyliczono przy pomocy wzoru:
    \begin{gather*}
        T=\frac{1}{\alpha}(\frac{R}{100[\Omega]}-1)
    \end{gather*}
    {\footnotesize
        \begin{itemize}
            \setlength\itemsep{0em}
            \item[] \textbf{T} - temperatura mierzona,
            \item[] \textbf{R} - rezystancja termopary,
            \item[] \textbf{$\alpha$} - współczynnik TWR termopary PT100 wynoszący $0.00385[K^{-1}]$,
        \end{itemize}}
    \noindent Przykładowo dla zmierzonej rezystancji termopary PT100 równej 118.901[$\Omega$]:
    \begin{gather*}
        T=\frac{1}{0.00385[{^\circ}C^{-1}]}(\frac{118.901[\Omega]}{100[\Omega]}-1)=49.094[{^\circ}C]
    \end{gather*}
    Charakterystyka temperatury obudowy rezystora HS50 od mocy wydzielonej na tym rezystorze została przedstawiona na wykresie nr 2 dołączonym do sprawozdania.
    \newpage
    \section{Uwagi i wnioski}
    Na podstawie wykresu nr 1 możemy zauważyć że charakterystyka zależności napięcia na termoparze od temperatury jest praktycznie liniowa. Pokrywałoby się to z
    charakterystykami napięciowo temperaturowymi większości termopar które na całym zakresie są bliskie bycia liniowymi.\\
    Na podstawie wykresu nr 2 możemy zauważyć że charakterystyka zależności temperatury obudowy HS50 od mocy wydzielonej na tym rezystorze nie jest liniowa
    a przypomina bardziej charakterystykę wielomianową stopnia 2. Wynika to z tego że wraz ze wzrostem temperatury na obudowie rezystora wzrasta różnica temperatur między
    obudową a otoczeniem. Powoduję to szybszą wymianą temperatury radiatora rezystora z otoczeniem. Obrazuje to zależność:
    \begin{gather*}
        Q=-kA\frac{\Delta T}{\Delta x}
    \end{gather*}
    {\footnotesize
        \begin{itemize}
            \setlength\itemsep{0em}
            \item[] \textbf{Q} - strumień ciepła/prędkość przepływu ciepła,
            \item[] \textbf{k} - przewodność cieplna materiału,
            \item[] \textbf{A} - pole powierzchni oddającej ciepło,
            \item[] \textbf{$\Delta T$} - różnica temperatury,
            \item[] \textbf{$\Delta x$} - grubość materiału przewodzącego ciepło,
        \end{itemize}}
    Wraz ze wzrastającą temperaturą, rosnąć będzie różnica temperatur co powoduje zwiększenie strumienia ciepła czego następstwem jest coraz większe tempo rośnięcia mocy
    pobieranej przez rezystor wraz ze wzrostem różnicy temperatury rezystora a temperatury otoczenia.\\
    \indent Dodatkowo można zauważyć że na wykresie nr 2 ostatni pomiar łamie powyższe przewidywanie. Jest to spowodowane zbyt krótkim czasem poświęconym
    na ustabilizowanie temperatury (wynikało to ze zbliżania się konica zajęć - eksperyment należało już przerwać).


    %Bibliografia
    \vfill
    \footnotesize
    \begin{thebibliography}{9}
        \bibitem{texbook1}
        https://wzn.pwr.edu.pl/materialy-dydaktyczne/metrologia-elektroniczna
        \bibitem{texbook3}
        https://en.wikipedia.org/wiki/Thermal\_conductivity\_and\_resistivity
        \bibitem{texbook4}
        https://en.wikipedia.org/wiki/Rate\_of\_heat\_flow
        \bibitem{texbook5}
        https://en.wikipedia.org/wiki/Temperature\_coefficient
        \bibitem{texbook6}
        https://www.dracal.com/en/what\-is\-a\-pt100\-rtd\-sensor
    \end{thebibliography}

\end{document}